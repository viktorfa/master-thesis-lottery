\section{Trustworthy computing}
\label{sec:trustworthy}

\subsection{Smart contracts}

Smart contracts is the concept of using computer code to enforce contract law. Because of the properties of digital signatures, ownership and transfer of value, property deeds, tokens, etc. can be enforced securely by computer programs. A smart contract is similar to any computer program in that it will execute some code, but it is different in that the execution and the result of it represent some legal concept, such as the right to vote, the ownership of something, or an identity. 

\subsection{Ethereum}

Ethereum \cite{wood2018ethereum} is a platform for for deploying and executing smart contracts. Ethereum has a virtual machine (EVM) with a Turing-complete instruction set and a key-value store which forms a global state. Transactions can be made to the virtual machine, which will execute some in a smart contract and alter the state of the machine. Transactions and the code they execute are processed in order by a global network of nodes called miners who run the EVM and process transactions by anyone who broadcasts them. Transactions are stored in blocks which form a blockchain which is considered immutable after some point in time. Because of this, Ethereum can serve as a global platform for enforcement of smart contracts where untrusting agents can conduct business, value transfer, governance, cooperation, and more on a large scale.

\subsubsection{Transactions}
Transactions are the basic unit of communication in the Ethereum system. Transactions can be seen as function calls to a specific smart contract. Miners validate each transaction before they include it to a block so that the state transition it executed is recorded. Such a state transition may represent cryptographic secure ownership of titles, funds, tokens and more, depending on the transaction itself and the participants involved in it. A smart contract, when invoked, will either execute its code and perform a state transaction or abort and not alter the state in any other way than transferring the gas included in the transaction to the account of the miner who mined that block. A special kind of transaction is the one which transfers ether from one account to another. It's different from other transactions in that it is not addressed to a deployed smart contract.
Although the instruction set of the EVM is Turing complete, programs are limited in what the can execute by some rules specified in the Ethereum yellow paper. Each operation and byte stored has a cost measured in gas. Gas is paid for with the ether currency, and a transaction is only valid if it has enough gas to run all its operations. While the price usually constitutes the limit on what transactions are made, there is also a limit to how much gas is allowed to use for a single transaction. This is because blocks are produced frequently (every 15 seconds at the time of writing), and a transaction which takes longer to executed than the expected time to produce blocks is not practical. 

\subsubsection{Accounts}
Accounts are agents in the Ethereum. All transactions must be initiated by an account, and each account has a balance of ether. There are two types of accounts: contract accounts (CA) and externally owned accounts (EOA). A contract account is a deployed smart contract and its code. A CA can receive and send ether and invoke other smart contracts, and its behaviour will depend entirely on its code. An EOA is addressed and represented by the public key of an asymmetric key pair. An EOA can receive and send ether and invoke smart contracts, but all transactions it makes must be signed with the private key of the account. 

\subsubsection{Ether}
Ether is a token native to Ethereum. It is limited in supply, and thus valuable. Each account has a balance denominated in ether. New ether is created with each new block as a block reward to the miner who produced the block. Gas, which is needed to run transactions, is paid for with ether, and each transaction must reserve some ether to be used for gas. Gas used in transactions is transferred to the miner who produce a transaction's block. Since ether is scarce and valuable, it can be used to incentivize participants in any application made in smart contracts. 