\section{Lotteries}
\label{sec:lotteries}

\todo{Summarize distributed lotteries. Can probably use lit review materials.}

Shamir et al. in \cite{shamir_mental_1981} introduced a protocol for two parties playing virtual card games that involve randomness without a trusted intermediary. Playing so called mental poker has been studied further in \cite{goldreich_how_1987} where the authors find a protocol to play any mental game that will work between any number of players at least as there is an honest majority. Lotteries are also mental games that require randomness, but since they typically have many participants and are susceptible to sybil attacks, traditional mental poker protocols won't easily work. 

Syverson in \cite{syverson_weakly_1998} and Goldschlag and Stubblebine in \cite{goldschlag_publicly_1998} suggested using data from sold tickets in a lottery as a source for entropy for a random function in order to create a lottery with a verifiable random process. A lottery where only internal information is used in the random process is called \emph{committed} or \emph{closed}. Both their protocols require some time to elapse between when the last ticket is bought – which finalizes the input to the random function – and when the result of the random function is available. Goldschlag and Stubblebine employ a delay function that takes a minimum time to calculate to achieve this, while Syverson include a weakly encrypted secret in each ticket that is moderately hard, but not infeasible, to decrypt and use to calculate the random function.

The concepts of a closed lottery and using delay functions in the random process have been used by a number of similar schemes in later contributions. Various enhancements such as privacy in \cite{zhou_playing_2001}