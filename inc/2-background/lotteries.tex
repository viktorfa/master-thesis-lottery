\section{Lotteries}
\label{sec:lotteries}

Shamir et al. in \cite{shamir_mental_1981} introduced a protocol for two parties playing virtual card games that involve randomness without a trusted intermediary. Playing so called mental poker has been studied further in \cite{goldreich_how_1987} where the authors find a protocol to play any mental game that will work between any number of players at least as there is an honest majority. Lotteries are also mental games that require randomness, but since they typically have many participants and are susceptible to sybil attacks, traditional mental poker protocols won't easily work.

Syverson in \cite{syverson_weakly_1998} and Goldschlag and Stubblebine in \cite{goldschlag_publicly_1998} suggested using data from sold tickets in a lottery as a source for entropy for a random function in order to create a lottery with a verifiable random process. A lottery where only internal information is used in the random process is called \emph{committed} or \emph{closed}. Such internal information could be a random number chosen by each ticket owner which is embedded in the ticket itself. Both their protocols require some time to elapse between when the last ticket is bought – which finalizes the input to the random function – and when the result of the random function is available. Goldschlag and Stubblebine employ a delay function that takes a minimum time to calculate to achieve this, while Syverson include a weakly encrypted secret in each ticket that is moderately hard, but not infeasible, to decrypt and use to calculate the random function.

The concepts of a closed lottery and using delay functions in the random process have been used by a number of similar schemes in later contributions. Various enhancements such as privacy in \cite{zhou_playing_2001} and more use of fair exchange in \cite{chow_e-lottery_2005}. A number of schemes have suggested using other means to generate randomness, such as verifiable random numbers generated by a committee of semi-trusted delegates in \cite{fouque_sharing_2001}, \cite{lee_design_2009}, \cite{liu_improved_2014}, and \cite{xia_information_2019}. Grumbach and Riemann in \cite{grumbach_distributed_2017} use a random process inspired by distributed voting schemes – a Kademlia distributed hash table (DHT) where each participant is a node who cooperates with other players in their subtree to generate verifiable randomness. 

Although these schemes can generate randomness for a lottery without a trusted intermediary, they don't provide equally trustless solutions to the other aspects of organizing a lottery. One issue is simply to have the participants agree on who are acutally participating. Most schemes solve this by having a lottery organizer digitally sign tickets so that the signature proves that a ticket is indeed valid. Should the lottery organizer refuse to pay the prize to a holder of a valid ticket, the ticket owner would have to complain to some other stakeholder. This can be solved by storing the prize money with a bank or payment processor who issues tickets and prizes based on fair exchange, but then we have a trusted intermediary involved. 
If the lottery is closed and information from the tickets are used as input to the random process, the organizer and participants need to agree on what is the correct set of tickets to be used. Since the organizer is the one who issues tickets, they could choose to not issue tickets during the last part of the purchase phase. Dong so the organizer can choose what input to use for the random function, and possibly calculate the delay function before the purchasing phase is finished. This issue is identified in the literature. One possible way of handling it is to require that the organizer publish the entire ledger of tickets continuously. It doesn't solve the issue completely, but it allows observers to see if suspicious behaviour is going on. Another solution is to have a trusted intermediary who can also issue tickets if the organizer is unresponsive to requests.

Syverson and Goldschlag et al. in \cite{syverson_weakly_1998} and \cite{goldschlag_temporarily_2010} discuss the topic of a pari-mutuel lottery concerning the dealer's ability to create tickets at will. If the lottery is pari-mutuel, meaning the entire prize pool comes from purchased tickets, the expected value of a ticket is independent on the number of tickets in existence, as the chance of winning decreases proportionally to the increased prize. But if the prize pool is larger than the cost of all tickets, the dealer would increase her expected value by forging tickets for themselves. Their conclusion is that a P2P needs to be pari-mutuel unless it can manage sybil behaviour by the dealer.

A number of lotteries have appeared on various blockchains. Due to the openness and verifiability of blockchain transactions, anyone with some coding skills can create a lottery or verify a lottery's fairness on such a platform. Although creating a completely fair lottery on a blockchain platform is non trivial, some claim to have accomplished the feat. SmartBillions \cite{noauthor_smartbillions_nodate} is a lottery on Ethereum that instantly pays out prizes. It's more akin to scratching tickets in that players play on their own without interacting with other players of the same game. One simply chooses a lucky number and finds out if one wins or not by the next block. The randomness does however come from a block hash which is not considered secure for large prizes, as it can be manipulated by miners \cite{bonneau2015bitcoin} \cite{pierrot_malleability_2018}. 
FairLotto \cite{ago_fairlotto_2018} is another lottery implemented on the Steemit blockchain. It works more as a typical lottery in that participants contribute to a prize pool by buying tickets, and one winner is selected by random to receive a share of the prize. It does however also rely on randomness from a source that can be manipulated by miners. It combines a secret the lottery organizer has committed to and the transaction id of the last ticket bought to generate a random number that will be used to select the winner. In such a scheme the organizer can potentially bribe miners to tactically select a transaction to be the last ticket, so that a ticket owned by the organizer will win. 
There are other lotteries that uses cryptocurrencies such as \cite{noauthor_satoshi_nodate}, but their randomness is completely generated by the lottery organizer, so it's actually just an online casino that uses cryptocurrencies, and not a distributed lottery protocol.

There has also been interest in distributed lotteries on the blockchain in the academic literature. Andrychowicz et al. proposed a lottery implemented in Bitcoin transactions in \cite{andrychowicz_secure_2014}. The authors identified argued that multiparty computation (MPC) on a blockchain can enforce honest behaviour by having participants make deposits that will be confiscated if they fail to follow the protocol of the computation. This can potentially solve the problem of previous lottery schemes that eventually have to rely on a trusted third party to enforce payments and correct behaviour by the lottery orgnanizer. A direct quotation from the introduction of Andrychowicz et al. paper follows to emphasize this point:

\begin{quotation}
  [MPC protocols] do not provide security
  beyond the trusted-party emulation. This drawback of the
  MPCs is rarely mentioned in the literature as it seems obvious
  that in most of the real-life applications cryptography cannot
  be “responsible” for controlling that the users provide the
  “real” input to the protocol and that they respect the output.
  Consider for example the marriage proposal problem: it is clear
  that even in the ideal model there is no technological way
  to ensure that the users honestly provide their input to the
  trusted party, i.e. nothing prevents one party, say Bob, to lie
  about his feelings, and to set b = 1 in order to learn Alice’s
  input a. Similarly, forcing both parties to respect the outcome
  of the protocol and indeed marry cannot be guaranteed in a
  cryptographic way. This problem is especially important in the
  gambling applications: even in the simplest “two-party lottery”
  example described above, there exists no cryptographic method
  to force the loser to transfer the money to the winner.
\end{quotation}

Following Andrychowicz et al., there has been designed a few more Bitcoin lotteries, including one concurrent by Bentov et al. \cite{bentov_how_2014}. These two schemes do however require a deposit that grows polynomially with the amount of participants. This makes a large lottery impractical, as prohibitively high deposits would be necessary to conduct the lottery securely. Bartoletti and Zunino in \cite{bartoletti_constant-deposit_2017} and Miller and Bentov in \cite{miller_zero-collateral_2017} independently designed similar lotteries that require only a constant or zero deposit. These lotteries work by constructing a tournament of digital coin tosses where each participant is paired with an opponent in $log_2(N-1)$ rounds. Half the participants are eliminated in each round until there is one winner left who can claim the prize. This scheme does a trade-off by minimizing deposits but increasing the number of rounds from being $O(1)$ to being $O(log(N))$. 
