\section{Blockchain}
\label{sec:blockchain}

A blockchain is an append-only data structure of blocks of structured data that are cryptographically linked to the previous block in a chain. Each block can contain a set of transactions that each alter a global state that is implicitly defined in all transactions in all blocks of the blockchain. Due to the links from a block to the previous block, one block cannot be altered without altering all subsequent blocks as well. If a blockchain is widely distributed and blocks are hard to produce, it becomes increasingly more difficult to alter a block the older it is. This gives blockchains an immutability property that makes it fit for purposes such as transferring and storing value.

A blockchain is commonly used as a global ledger that represents a state that defines ownership of various digital assets, these assets being primarily cryptocurrencies, but also property deeds and financial instruments. Blockchain systems are also typically permissionless and open for anyone regardless of their legal status or geographical location. A blockchain can be used for high value transfers and critical computations because it is considered to have the properties of immutability, liveness, and openness. 

\subsection{Transactions}

The purpose of a blockchain is to enable participants to reach a global consensus on an ordered list of transactions. Transactions are validated and embedded in blocks, and all transactions implicitly represent a state of the blockchain. In Bitcoin, the state that is represented is who has the authority to spend which coins. A transaction can specify a short script that makes some coins spendable by someone else.
The blockchain state can be represented as a set of accounts with balances denominated in bitcoin. Each transaction is applied to a state and results in an altered state. Transactions are validated before they are applied to the state. Although there exists many ways of validating different kinds of transactions, a transaction that spends coins that don't exist or spend more than it controls will never be validated. Usually, validation also involves verifying a digital signature.


\subsection{Mining}

Blocks are produced by miners who validate transactions and calculate a cryptographic puzzle that requires large amounts of computing power to be expanded. The puzzles work as a probabilistic process where miners need to map a block's data to a small subspace of a function's range. Such a mapping constitutes a valid proof-of-work (PoW) and allows the block to be appended to the blockchain. 

Miners compete to be the first to find such a PoW, and will be rewarded with some cryptocurrency if they produce a block that is accepted by the network. Since many miners are competing simultaneously, we will from time to time encounter a situation when two or more valid blocks are produced at about the same time. A miner who produces a valid block will broadcast it to other miners, who will in turn propagate it through the network. When a miners receives a valid block of height $n$ from a peer, they will start mining on top of that block to produce the block of height $n+1$. If several blocks are propagated through the network at about the same time, some miners will have different blocks of equal height, and will need to make a decision on which block to continue mining on top of. This situation is called a natural fork, as the chain is split at the end into more subchains. Eventually one of the subchains will be appended more than the other, and miners will accept whichever subchain that is longest. The shorter subchains will be abandoned and their transactions will not affect the state represented by the blockchain.


\subsection{Blockchain threats}

Since Bitcoin's inception there has been discussions around the security and threat models to blockchains. Much of the discussion involves the role of miners and their ability or inability to control the network. While there has been few unsuccessful attacks on the larger blockchains such as Bitcoin and Ethereum, the threat models and assumptions need to be reconsidered as blockchains find new use cases.

\subsubsection{Block reorganization}
During the situation of several subchains being on the network, one view of the blockchain might give a different state than another view. Say one observer had a view with subchain $s_1$ that included transaction $t_1$, and another observer had a view with subchain $s_2$ that did not include transaction $t_1$. If $s_1$ is eventually abandoned, as miners continued to mine on top of $s_2$, then $t_1$, which seemed to be included, is no longer part of the blockchain. This phenomenon is called a block reorganization.

A block reorganization violates the immutability property of the blockchain as blocks can be removed. Since miners have an interest in mining on the same chain, as only one chain's coins will be valuable, the network tends to reach consensus on which chain is correct quite quickly. A common heuristic for the Bitcoin blockchain is to consider transactions in blocks that have at least 6 blocks appended to them as immutable, which takes on average one hour to happen. As a general rule, the more blocks are mined on top of a block, the stronger its immutability. 


\subsubsection{Censorship}
Users can change the state of a blockchain by broadcasting transactions that will be included in a block by a miner. Due to limited space in blocks on blockchains like Bitcoin and limited computational resources on blockchains like Ethereum, there is a finite amount of transactions that will be included on the blockchain. Transactions can optionally include a fee that is paid to the miner who includes the transaction in their block, and this fee is the basis on which transactions miners choose to include. If miners are merely profit maximizing, such a pricing mechanism for blockchain resources maintains the liveness and openness properties of the blockchain. 
Miners do, however, have the power to discriminate transactions on any other basis as well. If a miner refuses to include transaction from a specific user, then that user will only have their transaction included if another miner includes it. If a collusion of miners all agree to not accept transactions from a specific user, if the collusion controls a large amount of the mining power, they will be able to censor that user. Such a situation will violate the openness property, as the blockchain is no longer accessible by anyone. 

\paragraph{Selfish mining}
Selfish mining is a type of miner behaviour that might allow censorship by a collusion with as little as 25\% of the mining power \cite{eyal2013majority}. While an altruistic miner will mine on top of whichever block they know about with the largest height, a selfish miner will bias their mining towards their own blocks. A selfish miner trying to mine a block of height $n$ will not immediately start to mine on the block of height $n+1$ when receiving a valid block of height $n$. Instead, they will continue trying to mine $b_n$ until they receive a block of height $n+k$ for some $k$, by when they start the same procedure again. 
If the selfish miner controls a small ratio of the mining power, such a strategy will lose potential profits, as they will waste resources on subchains that will be abandoned. As the ratio of their mining power goes up, their reluctance to accept blocks from other miners will cause the network to also abandon blocks from other miners at a more frequent rate than what the mining power would estimate. This will in turn incentivize other miners to more readily accept the selfish miner's blocks, as this will decrease the chances of their blocks being abandoned and the block reward being lost. 
If a selfish miner scenario plays out like that, the selfish miner can also choose to censor a certain type of transaction and force the network to join in on the censorship policy. The selfish miner enforces this by refusing to accept any block that contains a transaction of the type they censor. If the subchain the selfish miner favors is more likely to eventually be accepted, then other miners will produce blocks they are sure will be accepted by the selfish miner.

We note that the selfish miner scenario is a theoretical one. It assumes that altruistic miners are naive and profit maximizing, and will not react to the selfish miner's behaviour by ignoring them. The entire ecosystem around a blockchain is more complex than simply the software and consensus algorithms. Mining nodes and other participants are ultimately humans who can react and enact counter-measures to highly anti social behaviour by e.g. selfish miners. It is also generally assumed that miners have a stake in keeping the network's essential properties such as openness and immutability, since their profits are linked to the value of the cryptocurrency earned by mining the blockchain.
