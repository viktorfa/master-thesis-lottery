\section{Trustworthy computing}
\label{sec:trustworthy}

\subsection{Smart contracts}

Smart contracts is the concept of using computer code to enforce and secure contract law and specify other formal relationships and expectations in  society~\cite{szabo_formalizing_1997}. Digital signatures make it possible to authenticate an owner of digital assets. A system with an authentication mechanism and a record of digital assets that can be transferred between owners can serve as a smart contract platform. A smart contract is similar to any computer program in that it will execute some code, but it is different in that the execution and the result of it represent some legal concept, such as the right to vote, the ownership of something, an identity, or a more complex construct. A smart contract is not necessarily legal in that its code has legal status is any jurisdiction. It is rather that the smart contracts exist in an environment where they can enforce something, or the meaning of their computation represents something enforceable. In this computer environment, legal entities can be authenticated with digital signatures, and agreements, expectations and conditions can be written in code.

A simple example is a trust fund constructed in Bitcoin smart contracts. A grantor wants to make a trust fund for a beneficiary. The grantor programs a series of Bitcoin transactions $\{tx_0, tx_i, ..., tx_n\}$ so that each is spendable on the conditions that a) the transaction is digitally signed by the beneficiary, and b) $i$ years have passed since the trust fund was created. If the smart contract platform support sufficiently expressive scripts, it is easy to imagine more complicated constructs of a smart contract. The trust fund contract could be extended with almost arbitrary conditions, such as increasing the payment if a certificate of hospitalization or an acceptance letter from an educational institution is provided. In this way a legal construct can be created and enforced without the need for a code of law or a state apparatus to enforce the law.

\subsection{Ethereum}

Ethereum \cite{wood2018ethereum} is a platform for for deploying and executing smart contracts. Ethereum has a virtual machine (EVM) with a Turing-complete instruction set and a key-value store which forms a global state. Transactions can be made to the virtual machine, which will execute a smart contract and alter the state of the machine. Transactions and the code they execute are processed in order by a global network of mining nodes who run the EVM and process transactions by anyone who broadcasts them. Transactions are stored in blocks which form a blockchain which is considered immutable after some point in time. Because of this, Ethereum can serve as a global platform for enforcement of smart contracts where untrusting agents can conduct business, transfer value, do governance, engage in cooperation, and more on a global scale.

\subsubsection{Transactions}
Transactions are the basic units of communication in the Ethereum system. Transactions can be seen as function calls to a specific smart contract. Miners validate each transaction before they include it to a block where the transaction's code cause a state transition in the EVM. Such a state transition may represent cryptographic secure ownership of titles, funds, tokens and more, depending on the transaction itself and the participants involved in it. A smart contract, when invoked, will either execute its code and perform a state transition, or abort and not alter the state in any other way than transferring the gas included in the transaction to the account of the miner who mined that block. A special kind of transaction is the one that transfers ether from one account to another. It's different from other transactions in that it is not addressed to a deployed smart contract.

Even though the instruction set of the EVM is Turing complete, programs are limited in what they can execute by some rules specified in the Ethereum yellow paper. Each operation executed and byte stored has a cost measured in gas. Gas is paid for with the ether currency, and the result of a transaction's execution will be recorded only if it has enough gas to run all its operations. While the transaction cost usually constitutes the limit on what transactions are made, there is also a limit to how much gas is allowed to be used for a single transaction. This is because blocks are produced frequently \footnote{Blocks are mined on average every 15 seconds at the time of writing}, and a transaction that takes a long time to execute relative to the expected time to produce blocks is not practical. 

\subsubsection{Accounts}
Accounts are agents in the Ethereum network. All transactions must be initiated by an account, and each account has a balance of ether. There are two types of accounts: contract accounts (CA) and externally owned accounts (EOA). A contract account is a deployed smart contract and its code. A CA can receive and send ether and invoke other smart contracts, and its behaviour will depend entirely on its code. An EOA is addressed and represented by the public key of an asymmetric key pair. An EOA can receive and send ether and invoke smart contracts, but all transactions it makes must be signed with the private key of the account. 

\subsubsection{Ether}
Ether is a token native to Ethereum. It is limited in supply and necessary to make transactions, and thus valuable. Each account has a balance denominated in ether. New ether is created with each new block as a block reward to the miner who produced the block. Gas, which is needed to run transactions, is paid for with ether, and each transaction must reserve some ether to be used for gas. Gas is purchased with each transaction, and the amount used to execute a smart contract is transferred to the miner who produce a transaction's block. Since ether is scarce and valuable, it can be used to incentivize participants to act in certain ways in any application made in smart contracts.
