\section{Tournament lottery outline}
\label{sec:outline}

\subsection{Digital coin toss}

A coin toss is a random process in which a binary outcome is decided. While a physical coin toss is decided by a coin landing on either the heads or tails side, a digital coin toss is determined by a function with domain $\{0, 1\}$. In a coin toss in the usual sense, the two outcomes are equally likely. While either outcome can be biased to an almost arbitrary bias, all digital coin tosses in this thesis will have equally likely outcomes.

We consider the case where two untrusting parties want to arrive at an unpredictable outcome that is verifiable. One way of achieving this is using a two round protocol as described in \cite{blum1983coin} where each party commit to a hash in the first round, and then reveal the preimage in the second round. If we assume that a secure hash function is used, then either party can see the commitment of the other player without being able to guess what the preimage is. When the preimage, or secret, is revealed in the second round, either party can verify that the secret is actually the preimage of the commitment, thus making the protocol verifiable. The coin toss is performed by doing an operation, such as XOR, with both the parties' secrets as operands. The result of this operation is completely unpredictable to either party, as they do not know their counterparty's secret. For a 50-50 coin toss, we can simply decide the outcome by something simple as whether the least significant bit of the result is 1 or 0.

A secure implementation of this protocol would require the parties to mix in some salt when hashing the secret, i.e. by concatenating the secret with a predefined string. This is to prevent one party from committing to the same value as the counterparty, so that the preimages are equal to each other. That would result in the outcome being predictable, i.e. always $0$ if we use the XOR operation. 

Say we have Alice and Bob conducting a digital coin toss with commitments $c_A$ and $c_B$ and secrets $s_A$ and $s_B$. The salt is a predetermined string both parties use $salt$, and the hash function is $Hash()$. The commitments are calculated as $c:=Hash(salt, s)$, and are verified by the same operation. We get the result of the XOR operation $r=s_A \oplus s_B$. The outcome is then determined by if $r\ mod\ 2 = 0$, then Alice wins, or if $r\ mod\ 2 = 1$, then Bob wins.

\begin{figure}[htbp]
  \centering
  \includesvg[width=\columnwidth]{figures/digital_coin_toss_default_winner_flow.svg}
  \caption{Flowchart of a digital coin toss.}
  \label{fig:digital-coin-toss-flow}
\end{figure}

\subsubsection{Tournament of coin tosses}
A digital coin toss can determine a winner out of two players, but can we use the same mechanism to determine a winner when there are more than two participants? We can construct a tournament of matches where the winner of each match is determined by a digital coin toss. The tournament can be represented as a full binary tree where the root node is the final match, whose winner is the winner of the entire tournament. Winning any other match will make one advance to a match one step closer to the final match. The players of each match in an internal node, which we call \emph{internal matches}, is determined by the winners from each of its two children. Participants in the tournament are represented as an ordered set, and each participant is assigned as a player to exactly one leaf node match. 

Such a tournament will have $N=2^L$ players where $L$ is the height of the binary tree representing the tournament. There will be $\frac{N}{2}$ leaf node matches and $N-1$ matches in total. We refer to matches with the same height in the binary tree as belonging to the same \emph{level} in the tournament. Matches with the same level can be played concurrently and independently of each other, but matches in internal nodes of level $l$ cannot be played before all matches in level $l-1$ are determined.

It may be the case that one or both players in a match fail to complete the digital coin toss in a match. Matches of each level in the tournament will be initialized with global time limits for when players have to have completed a certain procedure, such as making a commitment or a reveal. If only one of the players fails to e.g. reveal their secret before the relevant time limit, that will be interpreted as a forfeiture, and the other player will win the match. If both players fail to complete the same procedure, e.g. if neither player makes a commitment before the commitment time limit, a default winner will be selected to ensure each match has a defined winner by the last time limit. The default winner behaviour makes sure that the lottery will always have a defined winner, but will not affect an honest player who follows the coin toss protocol.

\begin{figure}[htbp]
  \centering
  \includesvg[width=0.5\columnwidth]{figures/tournament_tree.svg}
  \caption{Tournament tree of 8 players.}
  \label{fig:tournament-tree}
\end{figure}

\subsection{Note on the lottery being non-deterministic}

\subsubsection{Why the tournament is not deterministic}  % Because of time outs
It's worth taking a moment to consider the implications of the timeout conditions in the tournament. Ideally, both players in each match will perform the digital coin toss. Since the secrets in the coin toss are committed to, if we assume both secrets will be revealed, there is just one possible outcome of the match. However, either one player fails to reveal their secret by the timeout, then the other player will win. The consequence of this being that a match is not deterministic when commitments are made. Only when both players have revealed their secrets or the timeout happens can the result be determined. 

This means that even if all players commit to secrets for each level of the tournament prior to playing a single match, any player can actually end up winning independently of the secrets, as the outcome can depend on who times out in which matches. Note that if we assume that all players will perform the coin toss successfully, then the tournament is completely deterministic once all commitments have been made, and there is just one single valid winner for that particular ordered set of players and commitments. If it were the case that no timeouts would happen, we could even use a single commitment and secret from each player that would be used for each match at every level of the tournament. If players can choose whether to reveal their secret or not, we cannot reuse secrets from one match to another, as will be demonstrated in the below.

\subsubsection{How to exploit a non-deterministic tournament}  % Why we need different commitments for each level
In a practical implementation of the tournament, we would have to have timeouts to avoid the protocol halting. We must also assume that players can be colluding. In a collusion, each player would know each other's secret, so they would know who would win against whom. If we simply used the same secret for matches in all levels, then even if the result is unpredictable before all players have revealed, a collusion could wait for all the non-colluding players to reveal, then tactically choose which players of the collusion should reveal, so that they are guaranteed to win matches in subsequent levels. Because of this, unless we can assume no collusion, we must require players to commit to different secrets in each match, and by that make the protocol more interactive.

\subsection{Phases}

We implement a lottery that uses a tournament of coin tosses to fairly determine the winner. It will be implemented on Ethereum where smart contracts are an important logical component. Our lottery is a system made out of smart contracts that interact is specific ways that will be explained below. The lottery has a lifecycle with separated phases, so it will be described according to those.

\subsubsection{Set up phase}
The tournament will be implemented as separate match contracts that each represent a match in the tournament tree. The matches are initialized with data that allows them to determine who its players are. Internal matches are initialized with two references to contracts in the previous level of the tournament, and it will call a \texttt{getWinner()} method from these contracts to determine who its players are. Leaf node matches, which we will call \emph{first level matches}, are initialized with an index that corresponds to a player in the ordered set of players. The leaf node matches have a reference to a \emph{master contract}, which they will call to retrieve the players that are assigned to the match.

While the match contracts form a tournament on their own, we need a contract to keep track of the state of the lottery and handle things such as taking deposits, maintaining the set of participants, and verifying who can claim the prize. This is the responsibility of the master contract. The master contract is initialized with the number of participants the lottery can have, which must be a power of 2, a price participants must pay in order to join, a start time $t_{start}$, and an end time $t_{final}$. To be able to verify the winner of the tournament, the master contract needs a reference to the final match contract. The entire lottery is set up by first deploying the master contract without a reference to the final match, as it has not been deployed yet. The match contracts will be deployed level for level, starting with the first level contracts. When the final match is deployed, the lottery will be initialized by setting the final match in the master contract. 


\subsubsection{Deposit phase}
Once the master contract and all the match contracts are set up, any potential player can verify that the contracts are set up correctly by inspecting each contract. Players can then join the lottery by sending a \texttt{deposit()} transaction that sends ether equal to the ticket price to the master contract. Making a deposit will increase the master contract's balance and insert the sender into a list of players, so that players are ordered by when they joined. In a lottery of $N$ players, the list of players will be indexed $i_{player}$ from $0$ to $N-1$. Each first level match is initialized with a unique index $i_{FLM}$ from $0$ to $\frac{N}{2}-1$. A first level match retrieves its players by looking up $i_{FLM} \cdot 2$ and $i_{FLM} \cdot 2 + 1$ in the match contract's list of players.

Deposits will be possible as long as the lottery is not full and a start time is not yet reached. The lottery is full once the amount of deposits has reached the predefined number of participants. If the lottery is not full by the start time, it will not be able to start, and players can withdraw their deposits. If the lottery is full by the start time, it moves into the next phase.

\subsubsection{Playing phase}
Each leaf node match contract should have a time $t_{commit}$ which is equal to the start time of the lottery, by when players can make commitments to the digital coin toss. The time limit $t_{reveal}$ is some time later when players can make a reveal transaction with their secret which is validated in the match smart contract. The last time limit of a match contract is $t_{play}$ by when it's no longer possible to reveal and the match is determined. When the match is determined, the method \texttt{getWinner()} will return a player address. If the current time is less than $t_{play}$, this method will raise an error. If there is a stalemate in the match, i.e. neither player makes a commitment and the $t_{play}$ limit is passed, the \texttt{getWinner()} method will return a default winner. The default winner is the first player in the match, i.e. the left player in the tournament tree. 

Internal match nodes should have a $t_{commit}$ equal to the $t_{play}$ limit of matches in the level below. As mentioned, each internal match has a reference to two matches from the previous level, in which the two winning players are eligible to play the internal match. This way the tournament proceeds level by level with time limits defined at contract initialization. Players will need to make transactions to the match contracts which validate the sender and the inputs. The final match will be played just like any other match, but the master contract will have a reference to this match, so that it can validate the winner of the tournament.

\subsubsection{Withdrawal phase}
Players can send a transaction to the \texttt{withdraw()} function of the master contract. If the final match has a winner, it will transfer the balance of the master contract to the sender, after validating that the sender is the same account as the winner. If the $t_{start}$ time limit has passed and the lottery is not full, the lottery did not get enough players to start. In this case, the \texttt{withdraw()} method will return the deposit to the sender if they have made a deposit.
