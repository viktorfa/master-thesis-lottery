\section{Security}
\label{sec:security}

Most applications on a blockchain will more or less inherit the blockchain's own security model. There is always a risk of censorship, loss of connectivity, block reorganization, and more, but open blockchains could still be the most secure computing platform we have for MPC without a trusted intermediary. A lottery is special in that the prize can be incredibly high, and for that reason it might attract more motivated attackers than applications that do not handle large amounts of value in a single transaction. By viewing security from an economics perspective, where an attack has a cost $c$, a chance of success $p$, and an exposed value $v$ that the attacker will gain if successful. If the expected reward $c \cdot p$ is so high that $c < v \cdot p$, the risks of an attack are too high. The profit potential for successful attack a large lottery can be so high that extraordinary measures can be taken by adversaries to succeed. We must therefore review below all the common security concerns such as censorship, network attacks, and block reorganizations when discussing the security of a lottery. 

\subsection{Loss of connectivity}
The lottery is inherently interactive in that participants need to make commitments and reveal secrets during the lottery playing phase. Since players who don't make a commitment or reveal within the time limits will lose, loss of connectivity is an issue. Since the lottery protocol requires interactivity, there is little we can do to mitigate this other than using longer time intervals between the steps ($t_{commit}$, $t_{reveal}$, $t_{play}$) in the match contracts, so that players have a chance to reconnect before a time limit is reached.

While players could mitigate against losing connectivity by outsourcing the interaction with the lottery to some third party service, that would entail sharing of secrets with a third party, which is a security consideration of its own. Players could run the lottery from servers under their control in data centers at different geographical locations than their web browser, and thus be quite safe against losing connectivity with minimal risk of leaking the secrets, but this is complicated for many users. 

\subsection{Blockchain reorganizations}

A block reorganization attack is a concern in that the results of a lottery can be reversed if a corrupt powerful miner is not happy with the result. Reversing the entire lottery, however, is probably not possible as block reorganization attacks are very expensive and difficult to perform on long subchains. A cheaper attack would be to wait for one's opponent in a match to reveal their secret, and then reorganize the blockchain if the result is not favorable. While it's certainly difficult to do so, we don't know exactly what the expected reward and cost would be for such an attack if the lottery prize is extremely high.

This concern can be mitigated by using longer time intervals between steps in the matches, as block reorganization attacks get more expensive the more blocks are involved. The production of blocks is quite inherent to the blockchain platform itself, and there is not much an application on the blockchain can do about that. One way of decreasing the risk of a block reorganization happening is using a blockchain that has a high mining power and much decentralization among miners, so that coordinating an attack is harder.

\subsection{Censorship and transaction blocking}

\subsubsection{Network attacks} While blocking of the network and eclipse attacks are relevant for all network applications, it is commonly assumed that long lasting attacks of this type will not happen. Even though there's always a risk of it happening, we can choose security parameters that minimize the consequences of such attacks. Again, with an interactive lottery, we can't do much more than to increase the time intervals between time limits in matches, so that the chance of getting one message through during that interval is high even when under attack. For targeted attacks on the network, players can also hide their location by accessing the network with privacy enhancing tools. Due to general network attacks being quite peripheral to the topic of this thesis, they are not discussed in detail, and the blockchain is assumed to be fairly resistant to these kinds of attacks.

\subsubsection{DoS attacks}
A denial-of-service attack (DoS) is done by flooding a network with bogus messages so that it is incapable of responding to legitimate messages. This can be done on a blockchain by broadcasting a large number of transactions with high transaction fees, so that miners will only include the bogus transactions in the blocks and ignore other transactions with normal transaction fees. Such an attack is costly to withhold over time, as transaction fees must be paid. But if the expected value of successfully blocking an opponent in a match in a lottery is high, it can very well be worth it. Such a transaction flooding attack can easily be countered by broadcasting a transaction with even higher transaction fees. This fact makes the relationship between attacker and defender asymmetric, as the defender only needs \emph{one} transaction to go through, while the attacker must block \emph{all} other transactions. A lottery client should take this into account and be ready to make transactions with high transaction fees if it suspects a DoS attack is under way.

\subsubsection{Censorship}
For a blockchain to have a high degree of liveness, i.e. transactions will be recorded rather quickly and reliably, we must assume that miners will include transactions by no other discrimination than the size of the transaction fees. However, miners are free to choose which transactions they include in the blocks they produce. They could for any reason refuse to include a transaction from or to a certain address, including from certain participants in a lottery. The main concern is an adversary paying miners bribes on the condition that transactions from certain participants are not included, or that miners themselves play in the lottery with the intent of abusing their power to censor. We see from how the tournament tree looks like that one would only need to block transactions from $log_2(N)$ participants to make sure that one wins each match by the other player failing to make a commit or reveal transaction. 

While it's likely that a non-corrupted miner will include transactions censored by other miners, and that the likelihood of that happening increases with time, corrupted miners can also choose to ignore blocks that are produced by honest miners, i.e. the selfish mining strategy. Such a situation could have the corrupted miner cause a block reorganization where blocks including censored transactions are replaced by the corrupt miner's blocks. Such a combination of censorship and block reorganization can be quite powerful if the corrupt miner or collusion of miners controls a large share of the mining power.

While a combined censorship and block reorganization attack backed by a large portion of the mining power is very hard to mitigate, less powerful censorship attacks can be mitigated in several ways. One is again to make sure the steps in the match contracts are sufficiently long. Another is to make the lottery less interactive and somehow reduce the amount of transactions that are necessary for completion. Another is to use anonymous transactions in which miners cannot know what the result of the transaction will be at mining time, but this cannot be accomplished without major changes to either the lottery design or the way mining on the blockchain works. The issue of censorship is discussed in more detail in Section~\ref{sec:censorship}.

\subsection{Compromised client and phishing}

\subsubsection{Compromised secrets}
If secrets are generated on the client such as a web browser, an adversary could trick participants to using a compromised client that leaks secrets to them. While this consideration is quite broad, as an attack would go through a player's computer and there is little we can do about a compromised computer when designing the lottery. Following standard practices for designing secure applications can be done in the application layer of the dApp.

\subsubsection{Compromised client}
Since it ss up to players to verify that the lottery is set up correctly by validating all the smart contracts, a compromised client could falsely validate a lottery that is set up in an adversary's favor. We assume that each player is capable of verifying that the lottery and tournament is set up correctly. An adversary could make it look like the lottery is set up correctly by luring players into a fake site with phishing. Verifying that the lottery is set up correctly is done by the client, as the smart contracts cannot do this on their own in our current design. E.g.~the reference to the final match contract is set in the master contract after initialization. The master contract does no more validation than verifying that the final match contract is a an Ethereum address. A malicious lottery organizer could make a bogus final match contract in which they are guaranteed to win, and which is not related to the actual tournament at all. Even if the final match is unfair, players can make deposits to the master contract, and the master contract will pay the prize to whoever is the winner of the match contract it was set up to use.

Perhaps having some sort of community in a chat room or a bulletin board for each lottery could mitigate this. Honest players could build up a reputation of trust and help other players, and suspicious lotteries could be reported and inspected by experts. Displaying a warning banner notifying players about the dangers of such attacks could also be a permanent fixture in the client application. But mitigating the concerns related to the client will not be discussed in more detail.
