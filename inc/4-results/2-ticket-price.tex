\section{Ticket price}
\label{sec:ticket-price}

Since setting up and playing the lottery requires transactions fees to be paid in ether, a lottery with ticket prices low relative to the transaction costs is not feasible. As we saw in Section \ref{sec:gas}, most of the transaction fees are paid by the lottery organizer. It is reasonable to assume that in a real implementation of the lottery, there would be support for the organizer to take a fee of the total prize – a \emph{house edge}, as a way to cover the costs for organizing the lottery. The fee should be at least so large that the transaction costs for deploying the contracts are covered. Since there is a risk that not enough players will join the lottery, and the organizer can claim no fee, it should be larger than only the transaction costs. 

For simplicity, we will in subsequent calculations assume that the organizer fee is equivalent to the transaction costs for setting up and playing the entire lottery. Even though the cost of playing the lottery is paid by participants and not the organizer, we use it in the calculations, as it is at least related to the transaction costs of the lottery. An alternative could be to speculate on a risk of the lottery not starting, and account for that in the organizer fee, but since it involves speculation, we prefer to use the concrete total transaction costs of the lottery.

Most common casino games have a house edge of about 1-10\%~\cite{walsh_houses_nodate}, while large lotteries often have a higher house edge of 40-50\%~\cite{shackleford_house_nodate}. We therefore consider the organizer taking a 10\% fee of the prize reasonable for both the players and the organizer. Since the prize is simply the sum of all the deposits, i.e. the income from ticket sales, and the transaction cost per participant is fixed, there is a lower bound on the ticket price if a 10\% fee is to cover the transaction costs.

By setting a ratio $r$ of what the organizer will take of the prize, and a gas price $gas\_price$, we can find a minimum ticket price $ticket\_price_{min}$ by using the gas usage $gas\_usage$ from Table \ref{tab:total-gas-usage-dual} for setting up the lottery per player. The expression is simply $ticket\_price_{min}=\frac{gas\_price \cdot gas\_usage}{r}$. If the ticket price is lower than this, the fee is not large enough to cover the lottery's transaction costs.
In our design, with a 10\% organizer fee and a 1.5 gwei gas price, this results in a minimum viable price of each ticket at about 0.03 ether, which at the time of writing is about 5 USD. That's a price we consider acceptable, even though the dollar price can change by at least an order of magnitude within months, as the ether price fluctuates a lot.

While the ticket price can be set arbitrarily high to make the necessary organizer fee arbitrarily low, as discussed in Section \ref{sec:censorship}, high prize could be a security concern. If we assume a maximum acceptable prize as a security parameter, the ticket price will also have an upper bound $ticket\_price_{max}$. This bound will be dependent on the organizer fee $r$ and the number of participants $N$. The prize of the lottery is expressed as $prize=(1-r)ticket\_price \cdot N$. If the prize is bounded to a maximum value for security reasons, it follows that the maximum acceptable ticket price will be $ticket\_price_{max}=\frac{prize_{max}}{N(1-r)}$. 

The upper bound on ticket price is dependent on both the amount of participants and the organizer fee. If we assume that the organizer fee will not be higher than about 50\%, we see that a maximum acceptable prize entails a maximum amount of participants, and is hence a scalability issue. The implications of this will be discussed further in Section \ref{sec:scalability}.
