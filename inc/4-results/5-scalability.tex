\section{Scalability}
\label{sec:scalability}

It's common for lotteries to have millions of participants. Their large scale is an important characteristic, so it's natural to explore the scalability of our implementation. While the Ethereum platform has no problem with having any amount of connections to participants all over the world, it has a limited amount of transactions that can be processed per time unit. Even if transaction costs can be overcome by making them low relative to ticket prices, the limit on transaction throughput will stay the same. The stakes involved in the lottery will also get higher the more participants join. This fact might create incentives for miners to censor transactions as described in Section~\ref{sec:censorship}. 
Both the limits on transaction throughput and the security exposure of a higher prize will put a limit on the scalability of the lottery. The former is a consequence of the limited resources on a global blockchain, while the latter is due to the possibility of an attack, and is thus a matter of security. The scalability of the lottery will be analyzed from both perspectives below.

\subsection{Transaction throughput}

Our lottery has clearly defined limits for the amount of transactions that need to be made for each participant. The amount of transactions of each type and its average gas usage is listed in Table~\ref{tab:gas-usage}. Setting up the lottery requires $N+1$ transactions, each participant joining takes one transaction for a total of $N$, playing each of the $N-1$ matches takes at most $4$ transactions each, and the winner needs a single transaction to withdraw the prize. So the maximum amount of transactions for a successful lottery will be $6N-2$.  

The time interval in which the transactions need to be made is important, as the intensity of transaction demand will vary during the lifecycle of the lottery. We expect a ticket purchasing period that lasts for several days where the demand for transactions will not be high per time unit. Right after the deposit phase, when the matches on the first level can be played, there are $\frac{N}{2}$ matches that all need to be interacted with at the same time. If the time interval between match phases of these matches is too low, the blockchain will not be able to handle all the transactions that need to be made. The demand for transactions after the first match will exponentially decrease for each level as the amount of matches is halved for each level in the tournament tree.

The amount of participants is $2^L$ and there are $2^{l-1}$ matches for each level if we count $l$ from $L$ at the first level matches to $1$ for the final match. The commit step and reveal step for each match require two transactions each. The amount of transactions needed for each step at each level will then be $txs\_step_{l}=2 \cdot 2^{l-1}=2^l$ – one for each remaining player. Ethereum has a transaction throughput capacity of roughly some amount of transactions per block $tpb$. The space for transactions in an Ethereum block is actually limited by the sum of gas used by its transactions. The commit transactions needed to play our lottery use on average approximately 80000 gas per transaction, which is the number we will operate with. Since blocks are mined on average at a fixed rate, transactions per block is equivalent to transactions per time unit.
Each step in a match lasts for a number of blocks $td$ which is set when each match contract is deployed. For it to be theoretically possible to perform all the transactions for each match, $td$ must be set so that $td > \frac{2^{l}}{tpb}$.

Ethereum currently has a capacity for about 100 transactions of our type per block~\footnote{As of 2019-05-07, the gas limit is 8000000 gas per block (\url{https://etherscan.io/}).}. Using this value for $tpb$, we can chart some values of what $td$ ought to be in the first level matches for various amounts of participants.

\begin{table}[h]
\centering
\caption{Estimates of $td$ if all transactions on the blockchain are used for our lottery.}
\label{tab:td-100percent-transactions}
\begin{tabular}{|l|l|l|l|l|l|}
\hline

L & N & td & td in s & td in m & td in h \\ \hline
8 & 256 & 3 & 45 & 0.8 & 0.01 \\ \hline
10 & 1024 & 11 & 165 & 2.8 & 0.05 \\ \hline
12 & 4096 & 41 & 615 & 10.3 & 0.17 \\ \hline
14 & 16384 & 164 & 2460 & 41.0 & 0.68 \\ \hline
16 & 65536 & 656 & 9840 & 164.0 & 2.73 \\ \hline
17 & 131072 & 1311 & 19665 & 327.8 & 5.46 \\ \hline
18 & 262144 & 2622 & 39330 & 655.5 & 10.93 \\ \hline
20 & 1048576 & 10486 & 157290 & 2621.5 & 43.69 \\ \hline

\end{tabular}
\end{table}

\begin{table}[h]
\centering
\caption{Estimates of $td$ if 10\% of the transactions on the blockchain are used for our lottery.}
\label{tab:td-10percent-transactions}
\begin{tabular}{|l|l|l|l|l|l|}
\hline

L & N & td & td in s & td in m & td in h \\ \hline
8 & 256 & 26 & 390 & 6.5 & 0.11 \\ \hline
10 & 1024 & 103 & 1545 & 25.8 & 0.43 \\ \hline
12 & 4096 & 410 & 6150 & 102.5 & 1.71 \\ \hline
14 & 16384 & 1639 & 24585 & 409.8 & 6.83 \\ \hline
16 & 65536 & 6554 & 98310 & 1638.5 & 27.31 \\ \hline
17 & 131072 & 13108 & 196620 & 3277.0 & 54.62 \\ \hline
18 & 262144 & 26215 & 393225 & 6553.8 & 109.23 \\ \hline
20 & 1048576 & 104858 & 1572870 & 26214.5 & 436.91 \\ \hline


\end{tabular}
\end{table}

\begin{table}[h]
\centering
\caption{Time for the entire lottery if 10\% of the transactions on the blockchain are used for our lottery.}
\label{tab:total-time-10percent-transactions}
\begin{tabular}{|l|l|l|l|l|l|}
\hline

L & N & total time & time in s & time in m & time in h \\ \hline
8 & 256 & 110 & 1650 & 27.5 & 0.46 \\ \hline
10 & 1024 & 420 & 6300 & 105.0 & 1.75 \\ \hline
12 & 4096 & 1650 & 24750 & 412.5 & 6.88 \\ \hline
14 & 16384 & 6568 & 98520 & 1642.0 & 27.37 \\ \hline
16 & 65536 & 26230 & 393450 & 6557.5 & 109.29 \\ \hline
17 & 131072 & 52446 & 786690 & 13111.5 & 218.53 \\ \hline
18 & 262144 & 104876 & 1573140 & 26219.0 & 436.98 \\ \hline
20 & 1048576 & 419450 & 6291750 & 104862.5 & 1 747.71 \\ \hline

\end{tabular}
\end{table}

The most realistic scenario is that only a small fraction of all the transactions in a block are used for the lottery. We also see that for a lottery of any size, there will probably be a quite high demand for transactions, which will in turn raise gas prices. This has implications for what the minimum viable ticket price should be, as it cannot be assumed that the gas price will be low for the first level of the lottery unless $td$ is very high. We that the bulk of transaction costs comes from deploying the lottery contracts, so thee increased gas price during the playing phase will have a limited impact on the total transaction costs.

Although $td$ can be set arbitrarily high, we probably don't want the lottery to drag on for weeks before a winner is determined. It seems the lottery will face scalability issues in the order of 100000s or from $2^{17}$ participants if it is to be completed within days. The number of transactions needed will decrease exponentially with higher levels, so the first two levels will account for 75\% of the total time of the playing phase. 

There are plans for Ethereum to increase the number of transactions it can handle by implementing proof-of-stake (PoS) and sharding. While this might make our lottery capable of handling more participants, we don't know whether such a change in the protocol might introduce a new threat model that makes the lottery less scalable in other ways.

\subsection{Transaction costs and max prize}

Due to the size of the prize increasing linearly with the number of participants, a censorship attack by miners gets more profitable the more participants there are, which puts a limit on scalability. Even if a censorship attack is unlikely to succeed, it cannot be ignored when it can give high rewards to dishonest miners. Since the lottery prize is decided by the ticket price as well as the number of participants, the ticket price can be set so low that the total prize is not high enough to encourage manipulation by miners. The ticket price, however, is bound to a minimum value where the transaction costs for playing the lottery get prohibitively high compared to the ticket cost. Using a gas price of 1.5 gwei and the gas usage for setting up and playing a dual match lottery found in Section~\ref{sec:gas}, we will present various charts to illustrate the scalability of our implementation of the lottery.

% Talk about the math used in the charts
The \emph{maximum prize} is a security parameter used in this analysis. A higher maximum prize exposes more value to be lost in case of a successful attack, and so the security risk increases with a higher maximum prize. The lottery organizer and participants can judge what their tolerance for a max prize is, and we will operate with values up to 3000 ether for illustration. Another important variable is the \emph{transaction cost ratio}. This is the ratio o  f ticket price to transaction costs for setting up and playing a lottery, i.e. $r=\frac{gas\_usage \cdot gas\_price}{ticket\_price}$. The prize is also entirely dependent on the ticket price, as the calculations for simplicity's sake assume the organizer takes a fee equal to the transaction cost ratio times the sum of all deposits, i.e. $prize=ticket\_price \cdot N(1-r)$ and $fee=ticket\_price \cdot Nr$ where $N$ is the number of participants. The ticket price is implicitly used in the charts, as any transaction cost ratio is mapped bijectively to a price when the transaction costs are fixed.

Figure~\ref{fig:cost-ratio-chart} shows the ratio of the ticket price going to transaction costs as a function of number of participants and a max tolerable prize. Figure \ref{fig:prize-chart} show the prize for the lottery for a given number of participants and ratio of transactions costs to ticket price.
The charts show us the maximum scale of the lottery under different conditions for maximum tolerable prize and ticket price going to transaction costs. With the assumptions that the transaction costs can be as much as 20\% of the ticket price, and the maximum prize being about 1500 ether, this scalability limit in terms of number of participants approaches that of the analysis of transaction throughput.

Figure~\ref{fig:price-chart} shows the transaction cost to ticket price ratio on the left y-axis which the thick blue line plots. The right y-axis shows the total prize for a given ticket price and is plotted by multiple thin lines that each represent a specific number of participants. The chart gives some idea of which tickets prices can be used for different amounts of participants and different maximum prizes. For instance, if a cost ratio of 0.1 is required, the thick blue line indicates that the ticket price will be 0.03 ETH. At that price, only lotteries with less than $32768=2^{15}$ will have a prize of less than 1000 ETH. If we start from the other direction and consider a lottery with $8192=2^{13}$ participants, plotted by the thin pink line, it can handle ticket prices of slightly more than 0.1 ETH if the prize is to be less than 1000 ETH, and a ticket price of 0.06 if the prize is to be below 500 ETH. Furthermore, it shows that lotteries with few participants can handle quite high ticket prices without risking too much exposure by a high prize, while lotteries with more than $2^{16}$ participants can barely handle ticket prices so low that the cost ratio is nearly prohibitive unless a very large prize is acceptable.

\subsection{Scalability limits}

It's difficult to say anything authoritative on the scalability in terms of maximum prize, as it is unknown how high a sensible maximum prize should be. If a large prize and organizer fee is acceptable, then the main scalability bottleneck will be in the transaction throughput of the Ethereum blockchain. If the transaction throughput were to be increased, or the number of necessary transactions to complete the lottery can be decreased, the bottleneck could be in the maximum acceptable prize. The discoveries in this section allows a lottery organizer to make informed choices on setting a ticket price, fee, and amount of participants, but do not help in figuring out how high a sensible max prize should be. We also see that the interactivity of the lottery design has consequences for the scalability in that it requires many transactions.

If the lottery playing phase is to be concluded with days and not weeks, and we assume that 10\% of all transactions on Ethereum can be related to the lottery for a while, the scalability limit of the current design is $2^{17}=131072$ participants. If we only consider ticket prices where the ticket price to transaction costs ratio is less than 0.2, $2^{17}$ participants is acceptable from a maximum prize perspective if the max prize is at least 1587 ether or 279312 USD. 

\begin{figure}[htbp]
  \centering
  \includegraphics[width=\columnwidth]{figures/max_participants_cost_ratio.png}
  \caption{Cost ratio as a function of participants and max prize.}
  \label{fig:cost-ratio-chart}
\end{figure}

\begin{figure}[htbp]
  \centering
  \includegraphics[width=\columnwidth]{figures/max_participants_prize.png}
  \caption{Prize as a function of participants and cost ratio.}
  \label{fig:prize-chart}
\end{figure}

\begin{figure}[htbp]
  \centering
  \includegraphics[width=\columnwidth]{figures/ticket_prices.png}
  \caption{Cost ratio and prize as a function of participants and ticket price.}
  \label{fig:price-chart}
\end{figure}
