\section{Cost of a censorship attack}
\label{sec:censorship}

The most concerning attack is probably one where miners tactically censor the transactions of one participant's opponents through the entire lottery. If a participant is not able to make their commit and reveal transactions, they lose that match. From the miners' perspective, a block reorganization attack is risky and expensive, but a censorship attack is almost costless. This means that if a miner is completely unconcerned with keeping the network healthy, even a marginal bribe would make it worth to censor someone. One could counter a small bribe by increasing the gas price in a transaction, so that the miner will be paid more in transaction costs than what the bribe is worth, but a briber would still have an advantage, as the prize of a successful attack would fund their activity. 

In practice, not many miners are willing to take bribes for censorship as they are either idealistic or have an interest in maintaining the reputation of the blockchain as a secure computing platform. At least the latter point is the case if miners have skin in the game of the blockchain, such as owning mining hardware specific for that chain or having large holdings of the currency of that blockchain. However, it's still useful to assess the feasibility of a censorship attack, as very large bribes can change incentives.

A lottery has $L$ levels in its tournament and $N=2^L$ players and a prize of $pr2^L$ where $r$ is the ratio of ticket deposits that go to the prize and $p$ is the price of participation. Ideally, each player has a $\frac{1}{2^L}$ chance of winning so that the expected value of participation is $\frac{pr2^L}{2^L}=pr$. For each round an adversary can successfully censor their opponent, they double their chance of winning and thus double their expected return. Note that the expected return doubles for each round, so that in the first round the expected return is $pr$, while in the last round it's $2^{L-1}pr$. So if bribing miners is barely worth it in one round, it's certainly worth it in the next round.

A censorship attack has a probability of succeeding for each block produced. If we assume there is not a powerful selfish miner that has the power to both censor transactions and ignore blocks that include the targeted transaction, there is a probability that an honest miner will include the targeted transaction. Since a coalition of selfish miners can be effective with as little as 30\% of the mining power, and certainly at 50\% of the mining power, we assume that in the worst case only 50\% of the miners are honest. If the ratio is any less than that, then censorship by selfish miners dominates the threat model. We see that even with just half of the miners being honest, the likelihood of a transaction being included in a block gets very high after just a dozen blocks. $p_{included} = 1-0.5^{12}=99.9756\%$.

We can measure the value of censoring a participant in one round by the increase in expected return for the adversary. As noted before, this value will increase exponentially as we get closer to the final match in the tournament. This also means that the value in large lotteries will be correspondingly large. If we assume a dollar value of a ticket be to 10 USD and we have $2^{16}=65536$ participants, the value of censoring your opponent in the last match will be $\frac{10 \cdot 2^{16}}{2}=327680$ USD, which at today's ether price makes it well worth it for a miner to risk losing some block rewards in order to get a chunk of the prize.

As the adversary's opponent broadcasts their reveal transaction to the network, the adversary and the colluding miners, or anyone with knowledge of the adversary's secret, will know what the result of the match will be. Since the match consists of a digital coin toss, in 50\% of the cases there will be no need for censorship at all for the adversary to win. The adversary can choose to censor only when they will not win honestly. This puts the honest participant at a disadvantage, as even though they can counter the bribe by making a transaction with a high gas price, they won't know whether they will win or not until the adversary has made their transaction.

This means that in the case of an honest player playing a match against an adversary engaging in censorship, the adversary would be willing to spend double the amount to bribe miners of what the honest player is willing to pay in transaction costs. An honest participant would only be willing to spend the entire expected value at that round, while the adversary is willing to spend double that if they know they would lose by playing honestly.

As to whether miners would accept such bribes, we don't know of any estimates of how large such a bribe must be for miners to collude in such a way. Assuming it is possible to bribe a powerful miner to censor transactions at all, the increasing possible gains of doing so increases as the lottery gets larger, so it will eventually be high enough. An opportunistic collusion of miners could also perform such an attack on their own.

Based on this analysis, we conclude that our proposed lottery is only secure under the assumption that there is no successful selfish miner behaviour on the blockchain. This is because a selfish miner with about a third of the mining power can in some cases effectively censor transactions, and our lottery requires transactions to be included within reasonable time. 

This can only be partly mitigated by having larger amounts of time between phases. But such a censorship attack is essentially costless and can be sustained indefinitely as long as the collusion of miners is powerful enough. We instead would recommend to not hold large lotteries or lotteries with very high prizes, and to be vary of powerful miners.
