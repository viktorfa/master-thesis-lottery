\chapter{Introduction}
\label{chap:introduction}

A lottery can be defined as a random distribution of a prize fund to a set of participants, or more generally, a random selection of a subset of winners from a larger set of participants to receive some privilege. A lottery is often associated with its use is gambling where one becomes a participant by contributing to a prize fund by buying a ticket, whereupon a set of winning tickets is randomly drawn and a part of the fund is redeemable by owning a winning ticket. Lotteries have also been used in important societal functions such as leader election in democratic governance \cite{sintomer_random_2010} or proof-of-stake systems, drafting of soldiers to war \cite{nixon_executive_1969}, jury selection, and distribution of scarce goods \cite{the_economist_why_2018}, or as a game used in conjunction with fundraising to a charitable cause. 

Since the stakes in a lottery can be quite high, as with large cash prizes or being drafted to a war, it is important that the result is unpredictable and unbiased. For the lottery to achieve its purpose of ultimately distributing the prize, it is important that there is consensus of the result, and that the rules are enforced. Achieving unbiased randomness and consensus is typically done in one of two ways: 1) auditing and public verification of every step of the protocol, i.e. paper tickets being sold and blindly drawn from a basket, as is common for small-scale charity lotteries, and 2) trust in a central authority to conduct the lottery fairly and according to the rules, which is typical for national lottos and government-initiated lotteries. Due to the large scale, auditing and verification of the second type are often done by a third party such as government or industry inspectors.

With the advent of the internet and peer-to-peer (P2P) online communications, it is natural to explore the possibility of porting lotteries to this domain. A lottery is a system of several components and is conducted in a process where a set of participants is defined, and a subset of winners is randomly chosen. Components of the system include a random process to decide the winners, an authentication process to decide the participants, a mechanism to distribute the prize, and often a way to handle payments.  A computerized lottery that is accessed over the internet can be made by having an authoritative server handle the entire lottery process. While this approach is not trivial and requires careful design \cite{sako_implementation_1999} \cite{konstantinou_electronic_2004} \cite{konstantinou_trust_2005} \cite{chen_design_2005} \cite{kuacharoen_design_2012} \cite{chen_novel_2016}, it assumes that the authoritative server or an auditor can be trusted to make sure the process is conducted correctly. Lotteries where we cannot assume trustworthy actors or a trusted intermediary will be referred to as \emph{distributed lotteries}.

In a distributed lottery, all the components of the lottery must work in a distributed network setting. It should be resistant to adversarial behaviour by participants such as sybil attacks or attempts to manipulate the random process, as well as a dishonest organizer. While there exist protocols to play games that involve randomness between non-trusting players \cite{shamir_mental_1981} \cite{blum1983coin} \cite{broder_provably_1985} \cite{goldreich_how_1987}, the scale of a lottery makes these protocols impractical. The scaling problem can be solved by making a random seed unpredictable with delay a function \cite{goldschlag_publicly_1998}, or by delegating the random process to a semi-trusted committee \cite{fouque_sharing_2001}. Despite good approaches to the random process, other components of a lottery, such as handling payments and enforcing the organizer to respect the result of the random process, are not easily solved. 

Blockchain platforms with cryptocurrencies mark an important shift in P2P computing, as they make it possible to transfer value and arrive at global consensus without a trusted intermediary. Due to scripting capabilities of several blockchain platforms, it is also possible to implement a number of financial instruments and games involving money. There have been gambling and lottery applications on Bitcoin and Ethereum for some time, and the topic of lotteries has been discussed in the academic literature. One class of lotteries uses digital coin tosses organized in a tournament to fairly select a winner. This scheme can in theory support a large amount of participants while also maintaining a high degree of resistance from collusion as well as verifiability of the entire lottery process.

A lottery of this sort has as far as we know been implemented as a proof-of-concept, but not been analyzed and discussed in detail. Applications of smart contracts have to cross a gap from theoretical soundness to practical feasibility. By analyzing an implementation in detail, measure its cost to deploy, and discuss its security and other issues, we hope to get a better understanding of lotteries on a blockchain.

\section{Thesis statement}
\label{sec:statement}

Miller and Bentov in \cite{miller_zero-collateral_2017} outline a fair lottery that can be implemented both on the Bitcoin platform and on other blockchains with a more expressive scripting language such as Ethereum. The authors note that an implementation of their lottery on Ethereum is significantly less complex and more scalable than the same scheme on Bitcoin. For this reason, as well as the fact that Ethereum has a rich ecosystem of developer tools and user interfaces that make smart contract applications accessible, we decided to implement a full version of the lottery on Ethereum. 

Although there exist several lotteries on Ethereum already, to our knowledge none fulfill the security requirements of Miller and Bentov's lottery. The purpose of creating an implementation is to get some insight into the viability of the lottery in a practical sense. While we will discuss theoretical considerations for the lottery, we expect to discover new considerations related to security, usability, and performance with a working implementation. 

What can we learn from implementing a lottery on Ethereum? What can we say about the security of such a lottery if implemented on a blockchain?

\section{Methodology}
\label{sec:methodology}

\subsection{Literature review}

A systematic literature review on p2p lotteries was conducted prior to writing this thesis in order to place the work in the context of published literature. Sources were collected by conducting a keyword search in online databases for published academic papers. The results from the search were restricted to the fields of computer and information science, as there were many irrelevant hits from different fields. All papers that included a design of a distributed lottery were included in the review. Additional relevant works were found in the reference list of papers found through the keyword search. 

Hits from the keyword search were first filtered based on their title and abstract. Papers that were clearly about a different topic than distributed lotteries were not included. Another round of filtering was done by reading the introduction, able of contents, and conclusion where papers that did not include a lottery design or something very similar were discarded. 

Both the Scopus and Webofscience databases were searched, but the relevant hits from Webofscience were a strict subset of the relevant hits from Scopus.

\begin{table}[h]
\centering
\caption{Keyword search on Scopus}
\begin{tabular}{|l|l|l|l|}
\hline

term & hits & filtered hits & lottery designs \\ \hline
Term 1 & 13 & 12 & 7 \\ \hline
Term 2 & 44 & 20 & 14 \\ \hline

\end{tabular}
\end{table}
\emph{Table of results from literature search. See terms below.}

\begin{itemize}
    \item Term 1: (bitcoin OR blockchain) AND (lottery OR lotteries) 
    \item Term 2: (verifiable OR verifiability OR p2p OR "peer-to-peer" OR distributed) AND (lottery or lotteries)
\end{itemize}

\begin{table}[h]
\centering
\caption{Designs of lottery schemes}
\begin{tabular}{|l|l|l|}
\hline

from keyword search & from references & total \\ \hline
19 & 6 & 25 \\ \hline

\end{tabular}
\end{table}

\subsection{Data collection}

A working implementation of a distributed lottery was implemented on a local version of the Ethereum blockchain. This is a proof-of-concept application that can be simulated and interacted with in order to collect data and get a sense of what a live implementation would look like. 
We collected data of transaction costs necessary to set up a lottery of various sizes by measuring the gas usage of simulations. 
We found the number of interactions a participant in the lottery is needed to have with the blockchain in order to complete the entire lottery process successfully. 
Data for gas price, ether price, and transaction throughput on the Ethereum blockchain was gathered from etherscan.com. 
Additional data from previous published papers are used when discussing results, and these will be clearly referenced.

\subsection{Data analysis}

The data collected from simulations were of high quality and did not need preprocessing. The data will be presented in \ref{chap:results}, and is used as the basis for analyzing certain properties of our lottery implementation. 

\section{Thesis structure}

\textbf{Chapter 2} introduces the reader to background material relevant to the topics discussed in the thesis. \newline
\textbf{Chapter 3} presents the implementation and design choices of the distributed lottery on Ethereum this thesis investigates.
\newline
\textbf{Chapter 4} is about the results generated from experiments with the implementation and an analysis of the viability and security of the implementation.
\newline
\textbf{Chapter 5} is a discussion on the the work and process that was used when doing the work for this thesis.
\newline
\textbf{Chapter 6} summarizes the thesis, contains the conclusion, and presents ideas for future work. 
